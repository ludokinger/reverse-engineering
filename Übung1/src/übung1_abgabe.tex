%%%%%%%%%%%%%%%%%%%
% Import Setting  %
%%%%%%%%%%%%%%%%%%%
\documentclass[
  paper=a4,
  fontsize=12pt,
  parskip=half,
  headheight=32pt,
  DIV=12,
  BCOR=3mm
]{scrartcl}

\usepackage[english, ngerman]{babel}
\usepackage[utf8]{inputenc}
\usepackage{amsmath}
\usepackage{graphicx}
\usepackage{bookmark}
\usepackage{array}
%usepackage{showframe} % to see the frame of the pdf
\usepackage[a4,center,cam]{crop} % set size of page frame. Synergy with Koma-Script

% Allows adding inline-code
\usepackage{listings}

% Allow direct copy&paste from the pdf
\usepackage[T1]{fontenc}
\usepackage[utf8]{inputenc}

\PassOptionsToPackage{colorlinks=true,linkcolor=blue}{hyperref}
%\usepackage[colorlinks=true,linkcolor=blue]{hyperref}

% Header and Footer
\usepackage[headsepline=0.4pt,plainfootsepline]{scrlayer-scrpage}
\pagestyle{scrheadings}
\clearpairofpagestyles

\ihead{Reverse Engineering}
\ohead{\includegraphics[width=2cm]{../img/HM_Logo.jpg}}
\usepackage{lastpage}
\cfoot{Seite \pagemark\hspace{0.9pt} von \hspace{0.9pt}\pageref{LastPage}}

\begin{document}

%%%%%%%%%%%%%%%
% Titelseite  %
%%%%%%%%%%%%%%%
\begin{titlepage}
    \newcommand{\HRule}{\rule{\linewidth}{0.35mm}} % Defines a new command for the horizontal lines

    \center% Center everything on the page

    \textsc{\LARGE Reverse Engineering}\\[0.7cm]

    \begin{figure} [!ht]
        \centering
        \includegraphics[width=5cm]{../img/HM_Logo.jpg}
    \end{figure}

    \textsc{\Large Übung 1 - Gruppe 3}\\[0.7cm]

    \HRule

    \begin{tabular}{*{3}{>{\centering}p{.25\textwidth}}}
        Ludwig Karpfinger & Armin Jeleskovic & Valentin Altemeyer \tabularnewline
        \url{ludwig.karpfinger@hm.edu} & \url{a.jeleskovic@hm.edu} & \url{valentin.altemeyer@hm.edu}
    \end{tabular}\par
    \HRule

    \textsc{}\\[0.3cm]

    \let\endtitlepage\relax

\end{titlepage}

%%%%%%%%%%%%%%%%%%%%%%%%%%%%%%
% Aufgaben ab hier einfügen  %
%%%%%%%%%%%%%%%%%%%%%%%%%%%%%%
\section*{Aufgabe 1 - REMnux installieren}
\section*{Aufgabe 2 - REMnux Tool Check}

\subsection*{a) Aus welchen Quellen kann ein Tool von REMnux stammen}
Remnux kann wie jede andere Linux-Distro Software aus den angegebenen Quellen installieren.
(abgesehen von sonstigen Paketmanagern, wie \textit{snap, flatpack, AppImage})
Folgender Befehl zeigt die Repos an:

\begin{lstlisting}[language=bash]
  $ sudo grep -Erh ^deb /etc/apt/sources.list*
\end{lstlisting}
Es fällt auf, dass ein spezielles Remnux Repo vorhanden ist namens:

\mbox{\textit{http://ppa.launchpad.net/remnux/stable/ubuntu}}. Diese Repo wurde durch \textit{remnux.sls} hinzugefügt\footnote{\url{https://github.com/REMnux/salt-states/blob/master/remnux/repos/remnux.sls}}

Die Besonderheit bei Remnux ist, dass der \textit{Remnux Installer} automatisch Software installiert, konfiguriert und aktualisiert.
Die Eigenschaften von Software, wie Download-Quelle, Installation Path, Hashnummer, Rechte, Abhängigkeiten und Configs, werden durch sogenannte \textit{salt-state Files} bestimmt.
Diese Files befinden sich auf GitHub und werden durch den \textit{Remnux Installer} geladen.

Remnux nutzt gemäß den Remnux Docs\footnote{\url{https://docs.remnux.org/behind-the-scenes/technologies/debian-packages}} folgende Installationsquellen:
\begin{itemize}
    \item pip
    \item gems
    \item npm
    \item apt Repos
\end{itemize}

\section*{Aufgabe 3 - File Classification}



\section*{Aufgabe 4 - Firmware Identifikation}



\end{document}