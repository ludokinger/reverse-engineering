%%%%%%%%%%%%%%%%%%%
% Import Setting  %
%%%%%%%%%%%%%%%%%%%
\documentclass[
  paper=a4,
  fontsize=12pt,
  parskip=half,
  headheight=32pt,
  DIV=12,
  BCOR=3mm
]{scrartcl}

\usepackage[english, ngerman]{babel}
\usepackage[utf8]{inputenc}
\usepackage{amsmath}
\usepackage{graphicx}
\usepackage{bookmark}
\usepackage{array}
%usepackage{showframe} % to see the frame of the pdf
\usepackage[a4,center,cam]{crop} % set size of page frame. Synergy with Koma-Script

% Allows adding inline-code
\usepackage{listings}

% Allow direct copy&paste from the pdf
\usepackage[T1]{fontenc}
\usepackage[utf8]{inputenc}

\PassOptionsToPackage{colorlinks=true,linkcolor=blue}{hyperref}
%\usepackage[colorlinks=true,linkcolor=blue]{hyperref}

% Header and Footer
\usepackage[headsepline=0.4pt,plainfootsepline]{scrlayer-scrpage}
\pagestyle{scrheadings}
\clearpairofpagestyles

\ihead{Reverse Engineering}
\ohead{\includegraphics[width=2cm]{../img/HM_Logo.jpg}}
\usepackage{lastpage}
\cfoot{Seite \pagemark\hspace{0.9pt} von \hspace{0.9pt}\pageref{LastPage}}

\begin{document}

%%%%%%%%%%%%%%%
% Titelseite  %
%%%%%%%%%%%%%%%
\begin{titlepage}
    \newcommand{\HRule}{\rule{\linewidth}{0.35mm}} % Defines a new command for the horizontal lines

    \center% Center everything on the page

    \textsc{\LARGE Reverse Engineering}\\[0.7cm]

    \begin{figure} [!ht]
        \centering
        \includegraphics[width=5cm]{../img/HM_Logo.jpg}
    \end{figure}

    \textsc{\Large Übung 1 - Gruppe 3}\\[0.7cm]

    \HRule

    \begin{tabular}{*{3}{>{\centering}p{.25\textwidth}}}
        Ludwig Karpfinger \tabularnewline
        \url{ludwig.karpfinger@hm.edu}
    \end{tabular}\par
    \HRule

    \textsc{}\\[0.3cm]

    \let\endtitlepage\relax

\end{titlepage}

%%%%%%%%%%%%%%%%%%%%%%%%%%%%%%
% Aufgaben ab hier einfügen  %
%%%%%%%%%%%%%%%%%%%%%%%%%%%%%%
\section*{Aufgabe 1 Unsigned LEB128 encoding}

LEB128 ist ein Encoding und wird zur compression benutzt.

High Bits sind wichtig wegen der Reihenfolge

\subsection*{1a) Unsigned LEB128 }

\begin{lstlisting}[]
    1. 543210 (dezimal) = 10000100100111101010 (binary) - convert
    2. 100001 0010011 1101010 - in 7er Gruppen aufteilen
    3. {0}100001 0010011 1101010 - Padding hinzuegen
    4. {0}0100001 {1}0010011 {1}1101010 - High Bits setzen
    5.   0x21        0x93       0xEA    - In Hexa umrechnen
    6. EA9321 - fertig
\end{lstlisting}

\subsection*{1b) ULEB128 to decimal}

\begin{lstlisting}[]
    1. d4e63a = 0xd4 0xe6 0x3a - Hex Umwandlung
    2. 0x3a 0xe6 0xd4 - sortieren
    3. 00111010 11100110 11010100 - umwandeln in Bits
    4. 0111010 1100110 1010100 - High Bits abziehen
    5. 963412 - umwandeln in dezimal
\end{lstlisting}

\subsection*{1c) Unsigned LEB128}

\begin{lstlisting}[]
    1. 186 (dezimal) = 10111010 (binary)
    2. 1 0111010
    3. 0000001 0111010
    4. 00000001 10111010
    5. 0x01      0xBA
    6. BA01
\end{lstlisting}

\section*{Aufgabe 2 Android Crackme}

\begin{enumerate}
    \item \textit{.apk file} in jadx-gui geöffnet.
    \item Manifest -> activity -> = Entry-Point
    \item Die Klasse \textit{MainActivity} wurde als Startpunkt ausfindig gemacht
    \item Flag \textit{5ecr3t} entdeckt
\end{enumerate}



\section*{Aufgabe 3 Assembler Programmierung}

\subsection*{3a)}

\subsection*{3b)}

\subsection*{3c)}


\section*{Aufgabe 4 Assembler Reversing}



\end{document}